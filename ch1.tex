%%% ch1.tex --- 

%% Author: jiangtao@tao-studio.net
%% Version: $Id: ch1.tex,v 0.0 2014-11-12 15:21:20 jiangtao Exp$


\chapter{行列式}
本章主要介绍$n$阶行列式的定义、性质及计算方法。此外还要介绍
用$n$阶行列式求解$n$元方程组的克拉默(Cramer)法则。
\section{二阶与三阶行列式}
\subsection{二元线性方程组与二阶行列式}
用消元法解二元线性方程组
\begin{equation}
  \label{equ:one}
\begin{cases}
  a_{11}x_1+a_{12}x_2 = b_1\\
  a_{21} x_1+a_{22}x_2 = b_2
\end{cases}
\end{equation}

为消去未知数\begin{math}x_2\end{math},以\begin{math}a_{22}\end{math}与\begin{math}a_{21}\end{math}
分别乘上列方程的两端,然后两个方程相减,得\\
\begin{equation*}
(a_{11}a_{22}-a_{12}a_{21})x_1=b_1a_{22}-a_{12}b_2
\end{equation*}
类似的,消去\begin{math}x_1\end{math},得
\begin{equation*}
(a_{11}a_{22}-a_{12}a_{21})x_2=a_{11}b_2-b_1a_{21}
\end{equation*}
当$a_{11}a_{22}-a_{12}a_{21}\neq0$时,求得方程组\ref{equ:one}的解为:
\begin{equation}
\label{equ:two}
x_1 = \frac{b_1a_{22}-a_{12}b_2}{a_{11}a_{22}-a_{12}a_{21}},x_2 = \frac{a_{11}b_2-b_1a_{21}}{a_{11}a_{22}-a_{12}a_{21}}
\end{equation}
\ref{equ:two}中的分子、分母都是四个数分两对相乘再相减而得。其中分母$a_{11}a_{22}-a_{12}a_{21}$是由方程组
\ref{equ:one}的四个系数确定,把这四个数按它们在方程组\ref{equ:one}中的位置,排行二行二列,(横排称行,
数排称列)的数表
\begin{equation}
  \label{equ:three}
\begin{array}{ccc}
  a_{11} & a_{12}\\
  a_{21} & a_{22}
\end{array},
\end{equation}
表达式$a_{11}a_{22}-a_{12}a_{21}$称为数表\ref{equ:three}的二阶行列式,并记作
\begin{equation}
  \label{equ:four}
  \begin{vmatrix}
  a_{11} & a_{12}\\
  a_{21} & a_{22}
\end{vmatrix}
\end{equation}
数$a_{ij}(i=1,2;j=1,2)$称为行列式\ref{equ:four}的元素,或元。元素$a_{ij}$的第一个下标$i$称为行标,
表明元素位于第$i$行,第二个下标$j$称为列标,表明元素位于第$j$列。位于第$i$行第$j$列的元素称为行
列式\ref{equ:four}的$(i,j)$元。

上述二阶行列式的定义,可以用对角线法则来记忆。参看图\ref{img:one},
\label{img:one} %% 此处插入图片
把$a_{11}$到$a_{22}$的实联线称为主对角线,把$a_{21}$到$a_{12}$的虚联线称为副对角线,于是二阶行列式便
是主对象线上的两元素之积减去副对角线上两元素之积所得的差。

利用二阶行列式的概念,\ref{equ:two}式中的$x_1$,$x_2$的分子也可以写成二阶行列式,即
\begin{equation*}
  b_1a_{22}-a_{12}b_2 =
  \begin{vmatrix}
    b_1 & a_{12}\\
    b_2 & a_{22}
  \end{vmatrix},
  a_{11}b_2 -b_1a_{21}=
  \begin{vmatrix}
    a_{12} & b_1\\
    a_{21} & b_2
  \end{vmatrix}
\end{equation*}

若记
\begin{equation*}
  D=
  \begin{vmatrix}
    a_{11} & a_{12}\\
    a_{21} & a_{22}
  \end{vmatrix},
  D_1=
  \begin{vmatrix}
    b_1 & a_{12}\\
    b_2 & a_{22}
  \end{vmatrix},
  D_2=
  \begin{vmatrix}
    a_{11} & b_1\\
    a_{21} & b_2
  \end{vmatrix},
\end{equation*}\\
那么\ref{equ:two}式可写成
\begin{equation*}
  x_1=\frac{D_1}{D}=\frac{
    \begin{vmatrix}
      b_1 & a_{12}\\
      b_2 & a_{22}
    \end{vmatrix}
  }{
    \begin{vmatrix}
      a_{11} & a_{12}\\
      a_{21} & a_{22}
    \end{vmatrix}
  },
  x_2=\frac{D_2}{D}=\frac{
    \begin{vmatrix}
      a_{11} & b_1\\
      a_{21} & b_2
  \end{vmatrix}}
  {
    \begin{vmatrix}
      a_{11} & a_{12}\\
      a_{21} & a_{22}
    \end{vmatrix}
  }。
\end{equation*}

注意这里的分母$D$是由方程组\ref{equ:one}的系数所确定的二阶行列式(称系数行列式),
$x_1$的分子$D_1$是用常数项$b_1$,$b_2$替换$D$中的$x_1$的系数$a_{11}$,$a_{21}$所确定的二阶行列式,
$x_2$的分子$D_2$是用常数项$b_1$,$b_2$替换$D$中的$x_2$的系数$a_{12}$,$a_{22}$所得的二阶行列式。  

例1  求解二元线性方程组
\begin{equation*}
  \begin{cases}
    3x_1-2x_2=12\\
    2x_1+x_2=1
  \end{cases}
\end{equation*}
解 由于\\
\begin{equation*}
  D=
  \begin{vmatrix}
    3 & -2\\
    2 & 1
  \end{vmatrix}
  =3-(-4)=7\neq 0\\
  D_1=
  \begin{vmatrix}
    12 & -2\\
    1 & 1
  \end{vmatrix}
  = 12 - (-2) = 14\\
  D_2=
  \begin{vmatrix}
    3 & 12\\
    2 & 1
  \end{vmatrix}
  = 3 - 24 = -21\\
\end{equation*}
\section{方程式}
\begin{equation*}
  a=b
\end{equation*}


\begin{equation}
  a_{11} x_1+a_{12}x_2 = b_1
  a_{21}x_1+a_{22}x_2 = b_2
\end{equation}

\begin{equation}
  a^2 + b^2 = c^2\\
  \frac{a}{b}+\frac2{33}=\frac{c}{d}
\end{equation}

\begin{equation} f(x) =
  \begin{cases}
    1 & \text{如果x为有理数} \\
    2 & \text{如果x为无理数} \\
    0 & \text{其它}
  \end{cases}
\end{equation}


